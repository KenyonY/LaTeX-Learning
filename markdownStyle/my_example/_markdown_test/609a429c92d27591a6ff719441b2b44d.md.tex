\markdownRendererHorizontalRule{}\markdownRendererInterblockSeparator
{}\markdownRendererHeadingThree{markdown风格的\LeTeX:}\markdownRendererInterblockSeparator
{}入门:\markdownRendererInterblockSeparator
{}\markdownRendererLink{本地文档}{file:///C:/Users/beidongjiedeguang/OneDrive/a_resume/marddown风格的latex/markdown/markdown.html#latex}{file:///C:/Users/beidongjiedeguang/OneDrive/a_resume/marddown风格的latex/markdown/markdown.html#latex}{}\markdownRendererInterblockSeparator
{}https://liam.page/2020/03/30/writing-manuscript-in-Markdown-and-typesetting-with-LaTeX/\markdownRendererInterblockSeparator
{}http://mirror.lzu.edu.cn/CTAN/macros/generic/markdown/markdown.pdf\markdownRendererInterblockSeparator
{}\markdownRendererInputFencedCode{./_markdown_test/d4668d680a809219873da29dd2d47405.verbatim}{bash}\markdownRendererInterblockSeparator
{}开启与LaTeX混合风格的输入模式:\markdownRendererInterblockSeparator
{}\markdownRendererInputFencedCode{./_markdown_test/b7e819acf5f115b7f04c7b6a0b924a0a.verbatim}{latex}\markdownRendererInterblockSeparator
{}这样,我们就可以在.md文件里写LaTeX代码了哈哈哈哈哈\markdownRendererInterblockSeparator
{}\markdownRendererHorizontalRule{}\markdownRendererInterblockSeparator
{}\markdownRendererUlBeginTight
\markdownRendererUlItem 自定义包 如果代码中自定义的命令和环境过多,可将它们打包放入单独的.sty文件中,单独维护管理:\markdownRendererUlItemEnd 
\markdownRendererUlEndTight \markdownRendererInterblockSeparator
{}\markdownRendererInputFencedCode{./_markdown_test/68a94bd6d02f2fd11d588fdaf80e3122.verbatim}{Tex}\markdownRendererInterblockSeparator
{}文件保持为\markdownRendererCodeSpan{mypkg.sty}, 然后就可以在.tex文件里调用了:\markdownRendererInterblockSeparator
{}\markdownRendererInputFencedCode{./_markdown_test/937859272bc53e8d1a92581bb317f566.verbatim}{tex}\relax