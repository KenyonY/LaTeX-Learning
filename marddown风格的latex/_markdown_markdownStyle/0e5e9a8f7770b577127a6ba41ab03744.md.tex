\markdownRendererHeadingOne{我是 Markdown}\markdownRendererInterblockSeparator
{}\markdownRendererBackslash{}LaTex\markdownRendererBackslash{} 这里可以用 Markdown 语法,撰写各种内容。例如,我可以\markdownRendererEmphasis{强调},也可以\markdownRendererStrongEmphasis{加粗},当然也可以\markdownRendererStrongEmphasis{\markdownRendererEmphasis{加粗并强调}}。\markdownRendererInterblockSeparator
{}\markdownRendererHeadingTwo{这里是二级标题}\markdownRendererInterblockSeparator
{}\markdownRendererBlockQuoteBegin
幸福的获得,在极大的程度上却是由于消除了对自我的过分关注。 ---Bertrand Arthur William Russell
\markdownRendererBlockQuoteEnd \markdownRendererInterblockSeparator
{}你看,我还可以使用引用↑。\markdownRendererInterblockSeparator
{}\markdownRendererHeadingTwo{关于 dash, en-dash 和 em-dash}\markdownRendererInterblockSeparator
{}LaTeX 使用者都应该知道 dash, en-dash 和 em-dash。dash 是普通的连字符,举例如:「five-year-old boy」。en-dash 是表示范围的稍长的横线,举例如:「以下章节是重点:12--15」。em-dash 则是英文中的破折号,举例如:「---Bertrand Arthur William Russell」\relax